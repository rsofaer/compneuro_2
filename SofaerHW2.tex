\documentclass[11pt]{modart}
\usepackage{geometry}                % See geometry.pdf to learn the layout options. There are lots.
\geometry{letterpaper}                   % ... or a4paper or a5paper or ... 
%\geometry{landscape}                % Activate for for rotated page geometry
%\usepackage[parfill]{parskip}    % Activate to begin paragraphs with an empty line rather than an indent
\usepackage{amsmath,amsfonts,amsthm,amssymb}
\usepackage{graphicx}
\usepackage{soul,color}
\usepackage{graphicx,float,wrapfig}
\usepackage{lastpage}
\usepackage{epstopdf}
\usepackage{fancyhdr}
\DeclareGraphicsRule{.tif}{png}{.png}{`convert #1 `dirname #1`/`basename #1 .tif`.png}

% Homework Specific Information
\newcommand{\hmwkTitle}{Homework 2}
\newcommand{\hmwkDueDate}{February 29, 2012}
\newcommand{\hmwkClass}{Computational Neuroscience}
\newcommand{\hmwkClassInstructor}{John Rinzel}
\newcommand{\hmwkAuthorName}{Raphael Sofaer}

% Setup the header and footer
\pagestyle{fancy}                                                       %
\lhead{\hmwkAuthorName}                                                 %
\rhead{\hmwkClass\ (\hmwkClassInstructor): \hmwkTitle}  %
\lfoot{\lastxmark}                                                      %
\cfoot{}                                                                %
\rfoot{Page\ \thepage\ of\ \pageref{LastPage}}                          %
\renewcommand\headrulewidth{0.4pt}                                      %
\renewcommand\footrulewidth{0.4pt}                                      %

\title{\large{\hmwkAuthorName}\vspace{0.1in}\\\textmd{\textbf{\hmwkClass:\ \hmwkTitle}}\\\normalsize\vspace{0.1in}\small{Due\ on\ \hmwkDueDate}\\\vspace{0.1in}\large{\textit{\hmwkClassInstructor}}\vspace{0.5in}}
\author{}
\date{}
  
\begin{document}
\maketitle

\section{2.1. Run the NIA tutorials.}
\subsection{Chattering Ion Channels}
\subsection{The Ca Action Potential}
\section{2.2. Summarize and expand on one tutorial.}
\section{2.3. Specific Questions:}
\subsection{From the “Chattering” Tutorial.  Estimate the conductance of a single Na+ 
channel and single K+ channel.  Describe how you did it}
\subsection{From the “Chattering” Tutorial.  No K+ channels; 200 Na+ channels.  For 
“testing”, use a 20 ms duration and 0 mV level.  For “conditioning” use a 20 ms 
duration and 3 different levels:  -65 mV, -80 mV, -95 mV.  Describe the features 
of Ina vs t for the 3 cases (plotted with 3 colors for 0<t<40ms, keep lines).  Why 
do curves differ if all were done with the same testing level?}
\subsection{Same as b but with 200 K+ channels and no Na+ channels}

\end{document}
